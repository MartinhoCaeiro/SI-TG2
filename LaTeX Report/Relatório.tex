\documentclass[conference]{IEEEtran}
\usepackage[utf8]{inputenc}
\usepackage{graphicx}


\title{Comparação de Algoritmos de Categorização \\ \large Categorizar um veículo a partir das suas informações \\
\textit{Comparation of Categorization Algorythms \\ \large Categorize a vehicle based on its information}}
\author{
\IEEEauthorblockN{Martinho Caeiro - 23917 || Paulo Abade - 23919}
\IEEEauthorblockA{
    Instituto Politécnico de Beja\\
    Escola Superior de Tecnologia e Gestão\\
    Beja, Portugal\\
    23917@stu.ipbeja.pt || 23919@stu.ipbeja.pt
}
}

\begin{document}
\maketitle
\begin{figure}[!ht]
    \centering
    \includegraphics[width=0.5\textwidth]{Resources/Logo/IPBejaESTIG.jpg}
\end{figure}

\begin{abstract}
Este artigo apresenta um estudo para a comparação entre algoritmos de categorização de veículos. O objetivo de cada um dos 
algoritmos é categorizar um veículo a partir das suas informações, sendo que este veículo será categorizado consoante o seu 
país de origem. Foi escolhido este tema para facilitar a nossa compreensão sobre o assunto e tornar mais agradável o estudo
destes algoritmos. O estudo foi realizado com base em algoritmos de aprendizagem supervisionada, nomeadamente o algoritmo
\textit{Binary Tree}, algoritmo \textit{Random Forest} TODO: Escolher mais algoritmos para fazer a comparação entre eles. 
Para fazer a comparação destes, foi utilizado um dataset, com cerca de 400 entradas, onde possui informações de veículos de 
diferentes países e este foi utilizado nos diferentes algoritmos para treinar e testar os mesmos. Este dataset, já foi alterado 
no primeiro trabalho da Disciplina de \textit{Sistemas de Informação}, sendo ligeiramente diferente, por já ter sido tratado. 
Os resultados obtidos foram comparados e analisados para perceber qual o algoritmo que melhor categoriza um veículo a partir das 
suas informações. A comparação dos algoritmos foi feita com base na sua precisão, sensibilidade e especificidade. TODO: Adicionar 
mais informações sobre o estudo realizado. Gráficos ROC?
\end{abstract}

\begin{IEEEkeywords}
algoritmos; veículos; categorização; aprendizagem supervisionada; árvore binária; random forest; precisão; sensibilidade; especificidade;
orange; datamining; machine learning; kaggle.
\end{IEEEkeywords}

\section{Introdução}
Neste artigo temos como objetivo comparar diferentes algoritmos de categorização de modo a entender qual o algoritmo mais eficiente para
a questão que está a ser feita. Para isso, foi utilizado um dataset com informações de veículos de diferentes países, onde o objetivo é
determinar o pais de origem dado as informações do veículo. Este dataset foi utilizado para treinar e testar os diferentes algoritmos de
categorização, nomeadamente os algoritmos \textit{Tree}, \textit{Random Forest}, \textit{Logistic Regression} e \textit{Neural Network}.

\section{Dataset}
O dataset "Car information dataset" \cite{ref1} utilizado neste estudo foi retirado do site \textit{Kaggle} e contém informações de 
veículos de diferentes países. Estas informações incluem a marca/modelo, a economia de combustivel, o número de cilindros, a cilindrada, 
a potência, o peso, a aceleração, o ano de fabrico e o país de origem. Este dataset possui cerca de 400 entradas em cada uma das colunas.

\section{Algoritmos de Decisão}
Nesta secção, vamos apresentar os diferentes algoritmos de decisão utilizados para a categorização dos veículos.

\subsection{Tree}
É um modelo baseado numa estrutura hierárquica em forma de árvore. Cada nó representa uma condição ou regra 
(geralmente um atributo do conjunto de dados), e os ramos dividem os dados com base nessa regra. O objetivo 
é chegar a uma decisão ou classificação no final de cada ramo (folha). É simples, interpretável e útil para 
problemas de classificação e regressão.

\subsection{Random Forest}
Este algoritmo é um conjunto de árvores de decisão. Cria várias árvores independentes, cada uma treinada com 
um subconjunto dos dados e das features (atributos) selecionados aleatoriamente. No final, combina os resultados 
(por votação, na classificação, ou pela média, na regressão) para melhorar a precisão e reduzir o risco de overfitting, 
comparado a uma única árvore.

\subsection{Logistic Regression}
Apesar do nome, é um método usado principalmente para classificação. Modela a probabilidade de um resultado pertencente a 
uma classe específica, usando uma função logística. É simples, rápido e eficaz em problemas de classificação binária, 
embora também possa ser estendido para múltiplas classes.

\subsection{Neural Network}
Inspiradas pelo cérebro humano, consistem em camadas de "neurónios" interligados. Cada neurónio recebe entradas, 
aplica uma ponderação e uma função de ativação, e passa o resultado para os neurónios da camada seguinte. São altamente versáteis 
e podem lidar com problemas complexos, como reconhecimento de imagens ou processamento de linguagem natural, mas requerem mais dados 
e poder computacional.

\newpage
%----------------------------------------------------------------------------------------------------------------------------------------------
\begin{table}[h!]
\caption{Comparação de Algoritmos de Decisão}
\centering
\begin{tabular}{|c|c|c|}
\hline
\textbf{Algoritmo de Decisão} & \textbf{Precisão (\%)} & \textbf{Tempo de Execução (s)}\\
\hline
Tree & xx & Conteúdo 2 \\
\hline
Random Forest & xx & Conteúdo 4 \\
\hline
Logistic Regression & xx & Conteúdo 4 \\
\hline
Neural Network & xx & Conteúdo 4 \\
\hline
\end{tabular}
\end{table}

\section{Conclusões}
Concluimos que a escolha do algoritmo correto é de extrema importância dado que a precisão, sensibilidade e especificidade
deve ser garantida e que a sua eficácia em grande escala tem um grande impacto no resultado final.

\section*{Trabalhos Relacionados}
Os trabalhos "Automobile EDA" \cite{ref2} e "EDA CAR INFORMATION DATA" \cite{ref3} são exemplos de trabalhos relacionados com este dataset.
Estes trabalhos tem outras abordagens e análises ao dataset, que tem como objetivo analisar as diferentes caracteristicas dos veículos.
 em vez de diferentes algoritmos, mesmo assim, são trabalhos foram úteis para uma compreensão mais aprofundada do dataset.

\begin{thebibliography}{1}
    \bibitem{ref1} T. Elmetwally, “Car information dataset”, \textit{Kaggle}, Maio 2023.
    \bibitem{ref2} A. Aboraida, “Automobile EDA”, \textit{Kaggle}, Setembro 2024.
    \bibitem{ref3} V. Salodkar, “EDA CAR INFORMATION DATA”, \textit{Kaggle}, Junho 2023.
\end{thebibliography}

\end{document}
