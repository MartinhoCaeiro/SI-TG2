\documentclass[conference]{IEEEtran}
\usepackage[utf8]{inputenc}
\usepackage{graphicx}

\title{Título do Artigo em Português \\ \large Sub-título se necessário em Português \\
\textit{Paper Title in English \\ \large Subtitle if needed in English}}
\author{
\IEEEauthorblockN{Nomes de Autores da 1a Instituição}
\IEEEauthorblockA{
    Linha 1 (da Instituição) Depart., Organização, Universidade\\
    Linha 2 (da Instituição)\\
    Linha 3: Cidade, País\\
    Linha 4: Endereços de Email
}
\and
\IEEEauthorblockN{Nomes de Autores da 2a Instituição}
\IEEEauthorblockA{
    Linha 1 (da Instituição) Depart., Organização, Universidade\\
    Linha 2 (da Instituição)\\
    Linha 3: Cidade, País\\
    Linha 4: Endereços de Email
}
}

\begin{document}
\maketitle

\begin{abstract}
Este documento electrónico é um template “vivo”. Os estilos dos vários componentes do artigo [título, texto, cabeçalhos, etc.] estão já definidos como ilustrado pelas diversas secções deste documento.
\end{abstract}

\begin{IEEEkeywords}
componente; formatação; estilo; títulos.
\end{IEEEkeywords}

\section{Introdução}
Este modelo fornece aos autores as principais especificações de formatação necessárias para preparar versões electrónicas dos seus artigos obedecendo ao modelo IEEE. Todos os componentes padrão do artigo foram especificados por três razões: (1) facilidade de uso para formatação de trabalhos individuais a serem incluídos em atas de conferências, (2) o cumprimento automático de requisitos electrónicos que facilitam a produção posterior de produtos electrónicos tais como atas de conferências digitais, e (3) a conformidade de estilo ao longo de um livro de atas de conferência.

\section{Facilidade de Utilização}

\subsection{Seleção do Template}
Em primeiro lugar confirme que tem o Template correto, formatado para o tamanho correto de papel (A4).

\subsection{Mantenha as Especificações Corretas}
O modelo é usado para formatar o seu artigo e definir os estilos do texto. Todas as margens, a largura das colunas, espaçamentos de linha e fontes de texto são definidos e não devem ser alterados em nenhum caso.

\section{Formatação Geral}
Nos artigos escritos em Português ou Espanhol coloque também título, resumo e palavras-chave em Inglês como indicado.

\subsection{Abreviaturas e Acrónimos}
Definir as abreviaturas e siglas a primeira vez que são utilizadas no texto, mesmo depois de terem sido definidas no resumo. Não use abreviatura no título ou títulos de secções a menos que sejam totalmente inevitáveis.

\subsection{Unidades}
\begin{itemize}
    \item Utilize unidades do Sistema Internacional (SI);
    \item Evite combinar unidades SI com outros sistemas (exemplo: corrente em amperes e campo magnético em oersteds) pois gera confusão;
    \item Use zero antes do ponto decimal: “0.25” e não “.25”.
\end{itemize}

\section{Equações}
As equações são uma exceção para as especificações deste modelo. Deve utilizar na equação as fontes Times New Roman e/ou Symbol. Numere as equações consecutivamente utilizando parênteses encostados à direita. Exemplo:

\begin{equation}
\alpha + \beta = \chi.
\end{equation}

\section{Figuras e Tabelas}
\subsection{Posicionamento}
Coloque as legendas das figuras por baixo das figuras e os títulos das tabelas em cima das tabelas. Insira as figuras e tabelas após serem citadas no texto. Utilize a abreviatura “Fig. 1” para se referir a figuras.

\begin{figure}[h!]
    \centering
    \includegraphics[width=0.5\textwidth]{IPBejaESTIG.jpg}
    \caption{Exemplo de uma legenda de figura.}
\end{figure}

\begin{table}[h!]
\caption{Título de uma Tabela Exemplo}
\centering
\begin{tabular}{|c|c|c|}
\hline
Cabeçalho de Coluna & Coluna 1 & Coluna 2 \\
\hline
Linha 1 & Conteúdo 1 & Conteúdo 2 \\
\hline
Linha 2 & Conteúdo 3 & Conteúdo 4 \\
\hline
\end{tabular}
\end{table}

\section{Conclusões}
O modelo IEEE e outros modelos disponíveis facilitam o trabalho de edição de atas de conferências e a produção de material científico. Utilize-o corretamente na escrita do seu artigo.

\section*{Agradecimentos}
Introduza agradecimentos a entidades e pessoas que contribuíram para o trabalho. Caso a submissão para a conferência seja \textit{Double-Blind}, introduza unicamente o texto \textbf{DFBR} (\textit{Deleted for Blind Review}).

\begin{thebibliography}{1}
\bibitem{ref1} G. Eason, B. Noble, and I. N. Sneddon, “On certain integrals of Lipschitz-Hankel type involving products of Bessel functions,” \textit{Phil. Trans. Roy. Soc. London}, vol. A247, pp. 529–551, April 1955.
\end{thebibliography}

\end{document}
